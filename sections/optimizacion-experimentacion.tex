\section{Optimización y experimentación} {En la siguiente sección, se evaluará el rendimiento de la base de datos mediante la ejecución de tres consultas complejas. Este análisis se llevará a cabo en varios escenarios que incluyen diferentes volúmenes de datos y se realizarán pruebas sin índices, únicamente con los índices por defecto y con los índices que consideremos más adecuados para garantizar la ejecución óptima de estas consultas. Finalmente, se procederá a analizar y comparar los resultados obtenidos.}
\subsection{Consultas SQL para el experimento}
\subsubsection{Descripción del tipo de consultas seleccionadas}
\subsubsection{Implementación de consultas en SQL}
\subsubsection*{Consulta 1}
\lstinputlisting[language=SQL, caption={Consulta 1}]{code-snippets/query_1.sql.}
\subsubsection*{Consulta 2}
\lstinputlisting[language=SQL, caption={Consulta 2}]{code-snippets/query_2.sql}
\subsubsection*{Consulta 3}
\lstinputlisting[language=SQL, caption={Consulta 3}]{code-snippets/query_3.sql}

\subsection{Metodología del experimento}

\subsection{Optimización de consultas}
\subsubsection{Planes de índices para Consulta 1}
\lstinputlisting[language=SQL]{code-snippets/query_1_index.sql}
\subsubsection{Planes de índices para Consulta 2}
\lstinputlisting[language=SQL]{code-snippets/query_2_index.sql}
\subsubsection{Planes de índices para Consulta 3}
\lstinputlisting[language=SQL]{code-snippets/query_3_index.sql}

\subsection{Plataforma de pruebas}

\subsection{Medición de tiempos}
\subsubsection{Sin índices}
\subsubsection{Con índices}

\subsection{Resultados}
\subsubsection{Consulta 1}
\subsubsection{Consulta 2}
\subsubsection{Consulta 3}

\section{Análisis y discusión}