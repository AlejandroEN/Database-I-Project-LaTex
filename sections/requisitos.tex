\section{Requisitos}
\subsection{Introducción}
En el mundo actual, las instituciones educativas son más que solo lugares de aprendizaje. Son microcosmos complejos que requieren una gestión eficiente y coordinada de una multitud de operaciones administrativas, académicas y operativas. Sin embargo, a medida que estas instituciones crecen y se vuelven más complejas, la gestión eficiente de estas operaciones se convierte en un desafío cada vez mayor.

Este desafío se ve agravado por la falta de sistemas de gestión eficientes que puedan manejar la creciente complejidad de las operaciones de las instituciones educativas. La falta de eficiencia y coordinación en la gestión de estas operaciones puede llevar a una serie de problemas, desde la mala gestión de los recursos hasta la insatisfacción de los estudiantes y los padres.

En este informe, abordaremos un problema específico que enfrentan muchas instituciones educativas: la falta de eficiencia y coordinación en la gestión administrativa, académica y operativa. Este problema no solo afecta la calidad de la educación que se ofrece, sino que también puede tener un impacto negativo en la reputación de la institución.

El objetivo de este proyecto es desarrollar una base de datos que pueda ayudar a abordar este problema. Con esto en mente, esperamos que este informe sirva como un punto de partida para entender la magnitud del problema y la necesidad de una solución efectiva. A medida que avanzamos desde una visión general de las instituciones educativas hasta un enfoque más específico en la necesidad de automatización, esperamos proporcionar una visión clara y detallada del problema que se resolverá.
\subsection{Descripción general del problema} (Ismael)
\subsection{Necesidad y usos de la base de datos} (Ismael)
\subsection{¿Cómo resuelve el problema hoy?} (Alessandra)
\subsubsection{¿Cómo se almacenan y procesan los datos?}
\subsubsection{Flujo de datos}
\subsection{Descripción detallada del sistema} (Alessandra)
\subsubsection{Objetos de información actuales}
\subsubsection{Características y funcionalidades esperadas}
\subsubsection{Tipos de usuarios existentes o necesarios}
\subsubsection{Tipo de consulta y actualizaciones} (Ismael)
\subsubsection{Tamaño estimado de la base de datos} (Ismael)
\subsection{Objetivos del proyectos} (Alessandra)
\subsection{Referencias del proyecto} (Alessandra)
\subsection{Eventualidades} (Ismael)
\subsubsection{Problemas que pudieran encontrarse en el proyecto}
\subsubsection{Límites y alcances del proyecto}