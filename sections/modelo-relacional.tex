\section{Modelo Relacional}
\subsection{Modelo Relacional}
\begin{itemize}
	\modeloRelacionalItem{\textcolor{darkGreen}{Institucion}}{\underline{\textcolor{blue}{ruc}}, \textcolor{blue}{descripcion}, \textcolor{blue}{fundador}, \textcolor{blue}{fundacionFecha}, \textcolor{blue}{bannerUrl}, \textcolor{blue}{nombre}}
	\modeloRelacionalItem{\textcolor{darkGreen}{Persona}}{\underline{\textcolor{blue}{dni}}, \textcolor{blue}{nombres}, \textcolor{blue}{primerApellido}, \textcolor{blue}{segundoApellido}, \textcolor{blue}{nacimientoFecha}, \textcolor{blue}{sexo}, \textcolor{blue}{email}}
	\modeloRelacionalItem{\textcolor{darkGreen}{Apoderado}}{\underline{\textcolor{blue}{\textcolor{darkGreen}{Persona}.dni}}, \textcolor{blue}{numeroCelular}}
	\modeloRelacionalItem{\textcolor{darkGreen}{Alumno}}{\underline{\textcolor{blue}{\textcolor{darkGreen}{Persona}.dni}}, \textcolor{blue}{\textcolor{darkGreen}{Salon}.nombreSeccion}, \textcolor{darkGreen}{Sede}.\textcolor{blue}{id}, \textcolor{darkGreen}{Apoderado}.\textcolor{blue}{dni}}
	\modeloRelacionalItem{\textcolor{darkGreen}{Colaborador}}{\underline{\textcolor{blue}{\textcolor{darkGreen}{Persona}.dni}}, \textcolor{blue}{sueldoHora}, \textcolor{blue}{cci}, \textcolor{blue}{numeroCelular}, \textcolor{blue}{horasSemanalesTrabajo}, \textcolor{blue}{estaActivo}}
	\modeloRelacionalItem{\textcolor{darkGreen}{Secretario}}{\underline{\textcolor{blue}{\textcolor{darkGreen}{Colaborador}.dni}}, \textcolor{darkGreen}{Sede}.\textcolor{blue}{id}}
	\modeloRelacionalItem{\textcolor{darkGreen}{Consejero}}{\underline{\textcolor{blue}{\textcolor{darkGreen}{Colaborador}.dni}}, \textcolor{darkGreen}{Sede}.\textcolor{blue}{id}}
	\modeloRelacionalItem{\textcolor{darkGreen}{Director}}{\underline{\textcolor{blue}{\textcolor{darkGreen}{Colaborador}.dni}}, \textcolor{darkGreen}{Sede}.\textcolor{blue}{id}}
	\begin{itemize}
        \item Relación One To One (1:1) con la entidad Sede, usamos el constraint UNIQUE en el traspaso hacia las tablas SQL, para asegurar que un Director solo pueda ser asignado a una sede.
    \end{itemize}
	\modeloRelacionalItem{\textcolor{darkGreen}{Tutor}}{\underline{\textcolor{blue}{\textcolor{darkGreen}{Colaborador}.dni}}, \textcolor{blue}{\textcolor{darkGreen}{Salon}.nombreSeccion}, \textcolor{darkGreen}{Sede}.\textcolor{blue}{id}}
	\begin{itemize}
        \item Relación One To One (1:1) con la entidad Salon, usamos el constraint UNIQUE en el traspaso hacia las tablas SQL, para asegurar que un Tutor solo pueda ser asignado a un salon.
    \end{itemize}
	\modeloRelacionalItem{\textcolor{darkGreen}{Profesor}}{\underline{\textcolor{blue}{\textcolor{darkGreen}{Colaborador}.dni}}}
	\modeloRelacionalItem{\textcolor{darkGreen}{ProfesorSede}}{\underline{\textcolor{blue}{\textcolor{darkGreen}{Profesor}.dni}}, \underline{\textcolor{darkGreen}{Sede}.\textcolor{blue}{id}}}
	\modeloRelacionalItem{\textcolor{darkGreen}{Sede}}{\underline{\textcolor{blue}{id}}, \textcolor{blue}{coordenadaLongitud}, \textcolor{blue}{coordenadaLatitud}, \textcolor{blue}{direccion}, \textcolor{blue}{construccionFecha}, \textcolor{darkGreen}{Institucion}.\textcolor{blue}{ruc}}
	\modeloRelacionalItem{\textcolor{darkGreen}{Grado}}{\underline{\textcolor{blue}{id}}, \textcolor{blue}{nombre}}
	\modeloRelacionalItem{\textcolor{darkGreen}{Curso}}{\underline{\textcolor{blue}{id}}, \textcolor{blue}{nombre}}
	\modeloRelacionalItem{\textcolor{darkGreen}{ProfesorCursoGrado}}{\underline{\textcolor{blue}{\textcolor{darkGreen}{Curso}.id}}, \underline{\textcolor{blue}{\textcolor{darkGreen}{Grado}.id}}, \underline{\textcolor{blue}{\textcolor{darkGreen}{Profesor}.dni}}, \textcolor{blue}{periodoAcademico}}
	\modeloRelacionalItem{\textcolor{darkGreen}{Salon}}{\underline{\textcolor{blue}{nombreSeccion}}, \underline{\textcolor{blue}{\textcolor{darkGreen}{Sede}.id}}, \textcolor{darkGreen}{Grado}.\textcolor{blue}{id}, \textcolor{darkGreen}{Tutor}.\textcolor{blue}{dni}, \textcolor{blue}{aforo}}
	\modeloRelacionalItem{\textcolor{darkGreen}{Matricula}}{\underline{\textcolor{blue}{year}}, \underline{\textcolor{blue}{\textcolor{darkGreen}{Alumno}.dni}}, \underline{\textcolor{blue}{\textcolor{darkGreen}{Sede}.id}}, \textcolor{darkGreen}{Grado}.\textcolor{blue}{id}, \textcolor{darkGreen}{Secretario}.\textcolor{blue}{dni}}
\end{itemize}
\subsection{Especificaciones de transformación}
\subsubsection{Entidades}
\begin{itemize}
	\item \textbf{Curso:} Se transforma en la tabla Curso con id como llave primaria. Cada curso tiene un nombre único. No hay dependencia directa con otras tablas a nivel de llave primaria.
	\item \textbf{Grado:} Se convierte en la tabla Grado con id como llave primaria, y nombre como atributo. Los grados organizan los cursos y se vinculan directamente con varios salones.
	\item \textbf{Sede:} Se transforma en la tabla Sede con id como llave primaria, además de tener al RUC de la institución como llave foránea. Incluye atributos como direccion, coordenadaLongitud, coordenadaLatitud, y construccionFecha. Representa una ubicación física donde se imparten cursos y trabajan los colaboradores.
	\item \textbf{Institucion:} Se convierte en la tabla Institucion con ruc como llave primaria. Incluye descripcion, fundador, fundacionFecha, bannerUrl, nombre. Esta entidad encapsula los datos fundamentales de la institución educativa.
\end{itemize}
\subsubsection{Entidades débiles}
\begin{itemize}
	\item \textbf{Salon:} Se convierte en la tabla Salon con una llave primaria compuesta por nombreSeccion y sedeId. Dependiente de Sede, reflejando que cada salón está ubicado en una sede específica. Atributos incluyen aforo, gradoId como foreign key.
	\item \textbf{Matricula:} La tabla Matricula define su llave primaria compuesta por alumnoDni, sedeId, y year, lo que refleja que un alumno se puede matricular en una sede específica cada año. Las claves foráneas incluyen gradoId y secretarioDni. gradoId vincula la matrícula al grado específico al cual el alumno está inscrito, facilitando la organización académica. secretarioDni conecta cada matrícula al secretario que procesó la inscripción, integrando la administración del proceso.
\end{itemize}
\subsubsection{Entidades superclases y subclases}
\begin{itemize}
	\item \textbf{Persona:} Superclase que se transforma en la tabla Persona con dni como llave primaria. Todos los individuos (alumnos, apoderados, colaboradores) se derivan de esta tabla, heredando dni y demás atributos personales.
	\item \textbf{Alumno, Apoderado, Colaborador:} Subclases de Persona. Cada una con sus respectivas tablas donde dni actúa como clave foránea y primaria. Alumno incluye nombreSeccion, sedeId y apoderadoDni, mostrando la dependencia y relaciones con otras entidades.
	\item \textbf{Profesor, Tutor, Secretario, Consejero, Director:} Subclases de Colaborador, cada una con roles y responsabilidades definidos, vinculados a sedes y otros elementos estructurales de la institución.
\end{itemize}
\subsubsection{Relaciones binarias}
\begin{itemize}
	\item \textbf{Salon y Sede:} Cada salón pertenece a una sede, representando una relación de 1 a n, donde cada sede puede contener varios salones.
	\item \textbf{Alumno y Salon:} Relación de n a 1, cada alumno está asignado a un salón específico.
	\item \textbf{Grado y Salon:} Relación de 1 a n, cada grado se imparte en varios salones, mostrando que un salón puede ser utilizado para diferentes grados dependiendo del horario y necesidad académica.
	\item \textbf{Profesor y Sede:} Esta relación binaria indica cómo los profesores están asignados a sedes específicas. La multiplicidad muestra que un profesor puede estar asignado a varias sedes, y cada sede puede tener múltiples profesores.
\end{itemize}
\subsubsection{Relaciones ternarias}
\begin{itemize}
	\item \textbf{ProfesorCursoGrado:} Esta relación muestra que los cursos son ofrecidos en varios grados por diferentes profesores. La multiplicidad aquí refleja que un curso puede ser impartido en varios grados y que múltiples profesores pueden enseñar el mismo curso en diferentes grados.
\end{itemize}
\subsection{Diccionario de datos}
\databaseTable{
	personaDni                & CHAR(8)               & X           & X           & DNI del alumno.                  \\
	\hline
	salonId                   & INT                   &             & X           & ID del salón asignado al alumno. \\
	\hline
	apoderadoDni              & CHAR(8)               &             & X           & DNI del apoderado del alumno.
}{Alumno}
\databaseTable{personaDni                & CHAR(8)               & X           & X           & DNI de la persona que es apoderado. \\
	\hline
	numeroCelular             & TEXT                  &             &             & Número de celular del apoderado.}{Apoderado}
\databaseTable{personaDni                & CHAR(8)               & X           & X           & DNI del colaborador.                              \\
	\hline
	sueldoMensual             & FLOAT8                &             &             & Sueldo mensual del colaborador.                   \\
	\hline
	cci                       & CHAR(20)              &             &             & CCI del colaborador para transacciones bancarias. \\
	\hline
	numeroCelular             & TEXT                  &             &             & Número de celular del colaborador.                \\
	\hline
	horasSemanalesTrabajo     & FLOAT8                &             &             & Horas de trabajo semanales del colaborador.}{Tabla Colaborador}
\begin{table}[H]
	\centering
	\begin{tabular}{|l|l|c|c|l|}
		\hline
		\textbf{Nombre del campo} & \textbf{Tipo de dato} & \textbf{PK} & \textbf{FK} & \textbf{Descripción}                              \\
		\hline
		colaboradorDni            & CHAR(8)               & X           & X           & DNI del consejero, que es un tipo de colaborador. \\
		\hline
		sedeId                    & INT                   &             & X           & ID de la sede donde trabaja el consejero.         \\
		\hline
	\end{tabular}
	\caption{Cuadro 7: Consejero}
\end{table}
\databaseTable{id                        & INT                   & X           &             & ID del curso.        \\
	\hline
	nombre                    & STRING                &             &             & Nombre del curso.}{Tabla Curso}
\input{tables/director-table.tex}
\databaseTable{id                        & INT                   & X           &             & ID del grado.        \\
	\hline
	nombre                    & STRING                &             &             & Nombre del grado.}{Tabla Grado}
\input{tables/institucion-table.tex}
\databaseTable{alumnoDni                 & CHAR(8)               & X           & X           & DNI del alumno matriculado.                     \\
		\hline
		year                      & DATE                  & X           &             & Año de la matrícula.                            \\
		\hline
		sedeId                    & INT                   & X           & X           & ID de la sede donde el alumno está matriculado. \\
		\hline
		gradoId                   & INT                   &             & X           & Grado en el que el alumno está matriculado.     \\
		\hline
		secretarioDni             & CHAR(8)               &             & X           & DNI del secretario que realizó la matrícula.}{Tabla Matricula}
\databaseTable{
	dni                       & CHAR(8)               & X           &             & DNI de la persona.                 \\
	\hline
	nombres                   & VARCHAR(100)                  &             &             & Nombres completos de la persona.   \\
	\hline
	primerApellido            & VARCHAR(50)                  &             &             & Primer apellido de la persona.     \\
	\hline
	segundoApellido           & VARCHAR(50)                  &             &             & Segundo apellido de la persona.    \\
	\hline
	nacimientoFecha           & DATE              &             &             & Fecha de nacimiento de la persona. \\
	\hline
	sexo                      & CHAR(1)                  &             &             & Sexo de la persona.                \\
	\hline
	email                     & VARCHAR(100)                  &             &             & Email de la persona.}{Persona}
\databaseTable{
	\hline
	Profesor.dni               & CHAR(8)               &            X & X           & DNI del profesor que imparte el curso.                  \\
	\hline
	Curso.id                   & INT                   &             X& X           & ID del curso que se imparte.                            \\
	\hline
	Grado.id                   & INT                   &             X& X           & ID del grado para el que se imparte el curso. \\
	\hline
	periodoAcademico & INT & & & Periodo acádemico en el que dicta el profesor.
}{ProfesorCursoGrado}
\databaseTable{
	Profesor.dni & CHAR(8) & X& X& DNI del profesor. \\
	\hline
	Sede.id & INT & & X& Sede en la que trabaja el profesor.
}{ProfesorSede}
\databaseTable{Colaborador.dni            & CHAR(8)               & X           & X           & DNI del profesor, que es un tipo de colaborador.}{Profesor}
\databaseTable{id                        & INT                   & X           &             & ID del salón.                                \\
	\hline
	aforo                     & INT                   &             &             & Capacidad máxima de estudiantes en el salón. \\
	\hline
	gradoId                   & INT                   &             & X           & ID del grado al que pertenece el salón.      \\
	\hline
	nombreDeSeccion           & TEXT                  &             &             & Nombre de la sección del salón.              \\
	\hline
	tutorDni                  & CHAR(8)               &             & X           & DNI del tutor asignado al salón.             \\
	\hline
	sedeId                    & INT                   &             & X           & ID de la sede a la que pertenece el salón.}{Tabla Salon}
\databaseTable{colaboradorDni            & CHAR(8)               & X           & X           & DNI del secretario, que es un tipo de colaborador. \\
		\hline
		sedeId                    & INT                   &             & X           & ID de la sede donde trabaja el secretario.}{Tabla Secretario}
\databaseTable{
	id                        & INT                   & X           &             & ID de la sede.                    \\
	\hline
	coordenadaLongitud        & DOUBLE                &             &             & Longitud geográfica de la sede.   \\
	\hline
	coordenadaLatitud         & DOUBLE                &             &             & Latitud geográfica de la sede.    \\
	\hline
	direccion                 & VARCHAR(255)                  &             &             & Dirección física de la sede.      \\
	\hline
	construccionFecha         & DATETIME              &             &             & Fecha de construcción de la sede. \\
	\hline
	Institucion.ruc               & CHAR(11)               &             & X           & RUC de la institucion.
}{Sede}
\databaseTable{
	Colaborador.dni & CHAR(8)               & X           & X           & DNI del tutor, que es un tipo de colaborador. \\
	\hline
	nombreDeSeccion           & VARCHAR(50)                  &         X    &             & Nombre de la sección del salón.              \\
	\hline
	Sede.id                    & INT                   &             & X           & ID de la sede donde trabaja el tutor.
}{Tutor}