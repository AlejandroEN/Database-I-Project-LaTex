\usepackage{relsize}
\usepackage{array}
\usepackage{tabularx}
\usepackage{imakeidx}
\usepackage{float}
\usepackage[T1]{fontenc}
\usepackage{graphics}
\usepackage{xcolor}
\usepackage{etoolbox}
\usepackage{graphicx}
\usepackage{rotating}
\usepackage{float}
\usepackage{listings}
\usepackage{xparse}

\colorlet{darkGreen}{green!40!black}

\lstdefinestyle{mystyle}{
	backgroundcolor=\color{gray!10},
	commentstyle=\color{blue},
	keywordstyle=\color{darkGreen},
	numberstyle=\tiny\color{gray},
	stringstyle=\color{red},
	identifierstyle=\color{blue},
	basicstyle=\ttfamily\footnotesize,
	breakatwhitespace=false,
	breaklines=true,
	captionpos=b,
	keepspaces=true,
	numbers=left,
	numbersep=5pt,
	showspaces=false,
	showstringspaces=false,
	showtabs=false,
	tabsize=2
}

\lstset{style=mystyle}

\newcommand{\databaseTable}[2]{
	\begin{table}[H]
		\centering
		\renewcommand{\arraystretch}{1.5}
		\begin{tabularx}{\textwidth}{|c|c|c|c|>{\centering\arraybackslash}X|}
			\hline
			\textbf{Nombre del campo} & \textbf{Tipo de dato} & \textbf{PK} & \textbf{FK} & \textbf{Descripción} \\
			\hline
			#1                                                                                                   \\
			\hline
		\end{tabularx}
		\caption{#2}
	\end{table}
}

\newcommand{\entidadBullets}[3]{
	\subsubsection{#1}
	\begin{sloppy}
		\begin{itemize}
			\item{Especificaciones: #2}
			\item{Consideraciones: #3}
		\end{itemize}
	\end{sloppy}
}

\newcommand{\modeloRelacionalItem}[2]{
	\item \textcolor{darkGreen}{#1} (#2)
}

\newcommand{\blue}[1]{\textcolor{blue}{#1}}

\newcommand{\green}[1]{\textcolor{darkGreen}{#1}}

\newcommand{\queryExecutionPlanItem}[2]{
	\begin{itemize}
		\item{#1:}
	\end{itemize}
	\begin{center}
		\includegraphics[width=\linewidth, height=0.75\textheight, keepaspectratio]{#2}
	\end{center}
}

\newcommand{\queryExecutionPlanGroup}[5]{
	\lstinputlisting[language=SQL]{#1}
	\queryExecutionPlanItem{Para mil datos}{#2}
	\queryExecutionPlanItem{Para diez mil datos}{#3}
	\queryExecutionPlanItem{Para cien mil datos}{#4}
	\queryExecutionPlanItem{Para un millón de datos}{#5}
}

\newcommand{\executionTimeTable}[2]{
	\begin{table}[H]
		\centering
		\renewcommand{\arraystretch}{1.5}
		\begin{tabularx}{\textwidth}{|c|c|c|c|>{\centering\arraybackslash}X|}
			\hline
			\textbf{Ejecución} & \textbf{1k} & \textbf{10k} & \textbf{100k} & \textbf{1m} \\
			\hline
			#1                                                                                                   \\
			\hline
		\end{tabularx}
		\caption{#2}
	\end{table}
}

