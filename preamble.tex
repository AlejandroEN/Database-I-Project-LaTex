\usepackage{relsize}
\usepackage{array}
\usepackage{tabularx}
\usepackage{imakeidx}
\usepackage{float}
\usepackage[T1]{fontenc}
\usepackage{graphics}
\usepackage{xcolor}
\usepackage{etoolbox}
\usepackage{graphicx}
\usepackage{rotating}
\usepackage{float}


\colorlet{darkGreen}{green!40!black}

\newcommand{\databaseTable}[2]{
	\begin{table}[H]
		\centering
		\renewcommand{\arraystretch}{1.5}
		\begin{tabularx}{\textwidth}{|c|c|c|c|>{\centering\arraybackslash}X|}
			\hline
			\textbf{Nombre del campo} & \textbf{Tipo de dato} & \textbf{PK} & \textbf{FK} & \textbf{Descripción} \\
			\hline
			#1                                                                                                   \\
			\hline
		\end{tabularx}
		\caption{#2}
	\end{table}
}

\newcommand{\entidadBullets}[3]
{
	\subsubsection{#1}
	\begin{sloppy}
		\begin{itemize}
			\item Especificaciones:
			      #2
			\item Consideraciones:
			      #3
		\end{itemize}
	\end{sloppy}
}

\newcommand{\modeloRelacionalItem}[2]
{
	\item \textcolor{darkGreen}{#1}(#2)
}

\newcommand{\blue}[1]{\textcolor{blue}{#1}}

\newcommand{\green}[1]{\textcolor{darkGreen}{#1}}